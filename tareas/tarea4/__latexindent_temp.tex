\documentclass[letterpaper,12pt]{article}
\usepackage[utf8]{inputenc}
\usepackage[spanish,es-tabla]{babel}
\decimalpoint
\usepackage{amsfonts}
\usepackage{float}
\usepackage{mathptmx}
\usepackage{booktabs}

%\usepackage{heuristica}
%\usepackage[heuristica,vvarbb,bigdelims]{newtxmath}

\usepackage[T1]{fontenc}
\renewcommand*\oldstylenums[1]{\textosf{#1}}
\usepackage[margin=1.3in]{geometry}
\usepackage{amsthm}
\usepackage{marvosym}
\usepackage{bm}

\renewcommand\qedsymbol{\Squarepipe}

\theoremstyle{definition}
\newtheorem{definition}{Definición}[section]
\newtheorem*{thm}{Teorema}


\setlength\parindent{0pt}

\newcounter{paragraphnumber}
\newcommand{\para}{%
  \vspace{10pt}\noindent{\bfseries\refstepcounter{paragraphnumber}\theparagraphnumber.\quad}%
}

%\setsecheadstyle{\large\bfseries}
%\setsubsecheadstyle{\bfseries}

\setlength\parindent{0pt}

\pagenumbering{gobble}


\usepackage{enumitem}
\setlist{nosep}

\usepackage{xcolor}

\usepackage{hyperref}
\hypersetup{
  colorlinks,
  linkcolor={red!50!black},
  citecolor={blue!50!black},
  urlcolor={green!50!black}
}

\usepackage{amssymb}
\usepackage{amsmath}

\begin{document}



\begin{center}
  {\large Aprendizaje Automatizado}\\
  \vspace{0.2cm}
  {\large\bfseries Tarea 4: Gráficos Probabilísticos}\\
  \vspace{0.2cm}
  {\large PCIC - UNAM}\\
  \vspace{0.5cm}
  {\itshape 20 de mayo de 2020}\\
  \vspace{0.5cm}
  Diego de Jesús Isla López\\
  (\href{mailto:dislalopez@gmail.com}{\itshape dislalopez@gmail.com})\\
  (\href{mailto:diego.isla@comunidad.unam.mx}{\itshape diego.isla@comunidad.unam.mx})\\
\end{center}


\section*{Ejercicio 1}

Se tienen las siguientes variables:

\begin{table}[H]
  \centering
  \begin{tabular}{|c|c|}
    \toprule
    Variable & Dominio \\
    \midrule
    \(A\) (alarma) & \{0,1\} \\
    \(D_A\) (alarma defectuosa) & \{0,1\} \\
    \(D_I\) (indicador defectuoso) & \{0,1\} \\
    \( I \) (lectura del indicador) & \(\mathbb{Z}\) \\
    \( T \) (temperatura real) & \(\mathbb{Z}\) \\
    \bottomrule
  \end{tabular}
\end{table}

\subsection*{Primer modelo}

La probabilidad conjunta del modelo queda expresada como:

\begin{equation}
  P(T,I,D_I,A,D_A) = P(T)\cdot P(I|T) \cdot P(A|I) \cdot P(D_I|T)\cdot P(A|D_A) \cdot P(D_A) 
\end{equation}

\subsection*{Segundo modelo}

La probabilidad conjunta del modelo queda expresada como:

\begin{equation}
  P(T,I,D_I,A,D_A) = P(T)\cdot P(I|T) \cdot P(A|I) \cdot P(D_I|T)\cdot P(A|D_A) \cdot P(D_A) 
\end{equation}

\subsection*{Tercer modelo}



\section*{Ejercicio 2}

\begin{itemize}
  \item \(T  \perp\!\!\!\perp F | D\)
\end{itemize}

\section*{Ejercicio 3}


\end{document}

